\chapter{Conclusion}

The Banner Summit Thermocouple Array is successful at measuring the temperature profile of a snowpack with an accuracy of \isostd \ (Figure \ref{fig:Iso_Temp_Hist}). This continuous temperature data allows us to conduct further analysis on the magnitude and duration of critical temperature gradients with an uncertainty of \gradstd \ (Figure \ref{fig:MC_Grad}). Although snowpack conditions can vary widely at the basin scale, the BSTA permits the construction of statistical relationships between a single site and nearby features such as stream gauges, or avalanche starting zones. A continuous snowpack temperature record, as derived via a temperature sensor array (as described here), will allow statistical analysis on avalanche hazards and snowmelt runoff. Further comparison between our results and snowpack energy balance models may provide insight into processes such as latent heat exchange and will help further our understanding of internal snowpack processes. 

Any time there is a phase change with subsequent migration of water molecules, the stable wate isotope concentrations of the remaining snow is altered. Measuring this change in stable isotope concentrations over time could improve our current understanding of internal snowpack processes. Future work should focus on reducing the error associated with sampling snow over a sizeable temporal domain. Improving our identification of specific storm layers will increase our ability to correlate between sampling events and will improve our ability to interpret this data. Additionally, establishing a snow sampling regime with a consistent datum, or ground surface, will reduce the error.   