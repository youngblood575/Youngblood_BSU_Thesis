\chapter{Conclusion}

The Banner Summit Thermocouple Array is successful at measuring the temperature profile of a snowpack with an accuracy of \isostd \ (Figure \ref{fig:Iso_Temp_Hist}). Critical temperature gradients were no longer present in the lower 20cm after the beginning of January. This continuous temperature data allows us to conduct further analysis on the magnitude and duration of critical temperature gradients with an uncertainty of \gradstd \ (Figure \ref{fig:MC_Grad}). In December of 2019, the lowest 20cm formed depth hoar and had above critical temperature gradients for around 50\% of the month. Critical temperature gradients were no longer present in the lower 20cm after the beginning of January. Temperature gradients in the upper 25cm can have a much larger magnitude, but the duration is controlled primarily by the diurnal solar cycle. Because this instrument precisely measures isothermal conditions in a snowpack, it may be possible to improve predictions of major snowmelt. Although snowpack conditions can vary widely at the basin scale, the BSTA permits the construction of statistical relationships between a single site and nearby features such as stream gauges, or avalanche starting zones. A continuous snowpack temperature record, as derived via a temperature sensor array (as described here), will allow statistical analysis on avalanche hazards and snowmelt runoff. Further comparison between our results and snowpack energy balance models may provide insight into processes such as latent heat exchange and will help further our understanding of internal snowpack processes. 

