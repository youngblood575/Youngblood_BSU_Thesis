\chapter{Discussion}

The Banner Summit Thermocouple Array is successful at measuring the temperature profile of a snowpack with an accuracy of \isostd \ (Figure \ref{fig:Iso_Temp_Hist}). This continuous temperature data allows us to conduct further analysis on the magnitude and duration of critical temperature gradients with an uncertainty of \gradstd \ (Figure \ref{fig:MC_Grad}). Although snowpack conditions can vary widely at the basin scale, the BSTA serves as a valuable tool because it may be possible to build statistical relationships between this site and things like nearby stream gauges, or avalanche starting zones. The ability to conduct statistical analysis for avalanche hazards and snowmelt runoff will only come with a continuous record from the BSTA at Banner Summit. Further comparison between our results and snowpack energy balance models may provide insight into processes such as latent heat exchange and will help further our understanding of internal snowpack processes. 

Stable water isotopes can improve our understanding of phase changes that occur within a snowpack. Future work should focus on reducing the error associated with sampling snow over a sizeable temporal domain. Improving our identification of specific storm layers will increase our ability to correlate between sampling events and will improve our ability to interpret this data. Additionally, establishing a snow sampling regime with a consistent datum, or ground surface, will reduce the error.   