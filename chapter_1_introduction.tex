\chapter{Introduction}
Snow in Idaho serves a critical role not only for recreationalists, but for the state economy. Most of the water that is used to irrigate crops in Idaho falls as snow in the winter months and is redistributed through a series of reservoirs and canal systems. In addition to this, snow and snowmelt can have major impacts on infrastructure and transportation throughout the state. By looking at certain snowpack characteristics, it’s possible to better predict things like avalanche hazards, floods caused by rain on snow events, and snowmelt runoff. 

In order to have considerable snowmelt, there are three basic phases a snowpack must go through (warming, ripening, and output). During the warming phase, any energy absorbed by the sub-freezing snowpack raises its average temperature until it reaches isothermal conditions at 0\textdegree C. This energy required to warm a snowpack to isothermal conditions is known as the cold content. Once the snowpack is isothermal, it enters the ripening phase where absorbed energy is used to melt snow, but the meltwater is retained in the snowpack by surface tension forces. After the snowpack is fully saturated and it is “ripe,” then it is in the output phase where further absorption of energy produces water output. Because isothermal conditions mark the beginning of the ripening phase, predicting snowmelt runoff timing can be done by measuring the temperature profile of a snowpack.

Snow cover in mid-to-high latitude areas serves a critical role in hydrological and meteorological processes. Because of this, there has been a strong push from the modeling community to better simulate the generation of spring snowmelt, the evolution and distribution of snow depth, and the formation and magnitude of snow load. In addition to this, a better understanding of surface boundary processes such as turbulent heat fluxes and heat conduction into and within the snow is critical in describing the lower boundary condition for atmospheric models.

The physical properties of snow also play an important role in avalanche prediction. Under the right conditions, temperature gradients within a snowpack drive vapor migration. As vapor migrates through and out of the snowpack, it changes the microstructure of the snow and can lead to a substantial water loss of 15\% - 20\% (\cite{hood_williams_cline_1999, marks_dozier_1992, kattelmann_elder_1991}). Continuous monitoring of snow temperature is valuable for avalanche forecasting because the development of weak, faceted layers depends on a temperature gradient. The critical temperature gradient to produce faceted forms in alpine snow is about 10\textdegree C/m (\cite{mcclung_schaerer_2009}). Consequently, a clear understanding of the conditions that drive these changes to the snowpack is important for both water resource management and avalanche safety.

Snow is a prime example of the observer effect; the mere observation of a phenomena within the snowpack inevitably changes the phenomena. The current method for measuring the temperature profile of a snowpack is a time consuming, destructive process that is not automated. The present method is destructive because it requires someone to dig a snow pit and manually measure the snow temperature. Not only does this disturb the snow profile and change its characteristics, but it’s also impossible to have high temporal resolution at remote alpine sites.

Here, we present a continuous snowpack temperature monitoring method coupled with analysis using stable water isotopes in order to provide insight to phase changes that occur within a seasonal snowpack. The temperature profile provides an estimate of the magnitude and duration of the forces driving phase change events as well as the timing of isothermal conditions. Developing methods for stable water isotope analysis will quantify the total flux of water vapor due to sublimation occurring in the snowpack.

