\chapter{Introduction}
Snow in Idaho serves a critical role not only for recreationists but for the state economy. A majority of irrigated agriculture in Idaho relies on surface water managed by a series of canals and reservoir systems. A majority of this water comes as snow during the winter months and is then stored in reservoirs for use during the growing season. Because of this, water managers in Idaho are actively looking for ways to better predict snowmelt runoff. Also, snow and snowmelt can have significant impacts on infrastructure and transportation throughout the state. By looking at specific snowpack characteristics, it's possible to predict things like avalanche hazards, floods caused by rain on snow events, and significant runoff. 

The physical properties of snow play an essential role in avalanche prediction. Under the right conditions, temperature gradients within a snowpack drive vapor migration. As vapor migrates through and out of the snowpack, it changes the snow's microstructure and can lead to a substantial water loss of 15\% - 20\% (\cite{hood_williams_cline_1999, marks_dozier_1992, kattelmann_elder_1991}). Continuous monitoring of snowpack temperature gradients is valuable for avalanche forecasting because the development of weak, faceted layers depends on a temperature gradient. The critical temperature gradient to produce faceted forms in alpine snow is about 10\textdegree C/m (\cite{mcclung_schaerer_2009}). 

At Banner Summit in Idaho, a significant concern for avalanche forecasters is in the development of depth hoar. Depth hoar is large-grained, faceted, cup-shaped crystals near the ground and forms because of large temperature gradients within the snowpack (\cite{akitaya1974}). This phenomenon most commonly happens in the early season because the snowpack is shallow, and there isn't much snow insulating the lowest layers from the cold atmosphere. In central Idaho, the duration and magnitude of critical temperature gradients in the lower snowpack is not well understood. 

Avalanches are primarily a concern in the early to mid-winter season. As spring approaches, a snowpack temperature profile provides insight into the snowmelt process. There are are three primary phases a snowpack must go through to have considerable melt and runoff; warming, ripening, and output (\cite{dingman2015}). Any energy absorbed by the sub-freezing snowpack during the warming phase raises its average temperature until it reaches isothermal conditions at 0\textdegree C (\cite{dingman2015}). This energy required to warm a snowpack to isothermal conditions is known as the cold content (\cite{dingman2015}). Once the snowpack is isothermal, it enters the ripening phase where absorbed energy melts snow, but the meltwater is retained in the snowpack by surface tension forces (\cite{dingman2015}). After the snowpack reaches its water holding capacity and it is "ripe," then it is in the output phase where further absorption of energy produces water output (\cite{dingman2015}). Because isothermal conditions mark the beginning of the ripening phase, it may be possible to predict snowmelt runoff timing by measuring the snowpack's temperature profile.

Snow is a prime example of the observer effect; the mere observation of a phenomena within the snowpack inevitably changes the phenomena. The current method for measuring the temperature profile of a snowpack is a time consuming, destructive process that is not automated. The present method is destructive because it requires someone to dig a snow pit and manually measure the snow temperature. Not only does this disturb the snow profile and change its characteristics, but it’s also impossible to have high temporal resolution at remote alpine sites.

Here, we present a continuous snowpack temperature monitoring system along with methods for analyzing the data. Results suggest this instrument is successful at measuring the temperature profile of a snowpack. This data further develops our understanding of temperature gradient metamorphism in snow, and it provides insight into snowpack processes that lead to significant snowmelt.

