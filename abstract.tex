\chapter*{Abstract}
\addcontentsline{toc}{chapter}{Abstract}

Predicting metamorphism within seasonal snowpacks is critical for avalanche forecasting and runoff timing as it relates to water supply management. Snowpack temperature gradients play a key role in snow metamorphism, and their magnitude controls how snow strength changes; therefore, they are of interest to avalanche forecasters. Before major melt, the snowpack must warm to isothermal conditions at 0\textdegree C. Measuring this transition from warming to the ripening phase could help improve our current models for runoff timing. Measuring snowpack temperature gradients is currently a non-automated process that requires disturbance of the snow profile, and only gives a snapshot in time of the temperature conditions. Here we demonstrate an automated method to monitor in situ snowpack temperature using a thermocouple array, co-located with the Banner Summit SNOTEL site in central Idaho. Showing the location and duration of critical temperature gradients helps avalanche forecasters detect warning signs related to possible facet formation. During the 2019 winter, we observed large temperature gradients in the bottom 20cm of the snowpack, with the gradient falling below critical (\textless 0.1\textdegree C/cm) by early January. Critical gradients were observed near the surface throughout the winter, and temperatures were within \isostd \ of the melting point when the snowpack became isothermal in the spring. We anticipate this dataset will inform snowpack energy balance models and aid in the prediction of avalanche hazards and runoff timing.

